\documentclass{article}
\usepackage{hyperref}
\usepackage{listings}
\usepackage{color}

% Define custom colors for code
\definecolor{codegray}{gray}{0.9}
\definecolor{codeblue}{rgb}{0,0,0.6}

\lstset{
  backgroundcolor=\color{codegray},
  basicstyle=\ttfamily\small,
  keywordstyle=\color{codeblue},
  breaklines=true,
  frame=single,
  showstringspaces=false
}

\title{Guia Detalhado de Parametrização do Comando FFmpeg para Uso com GPU (CUDA) e CPU}
\author{Seu Nome}
\date{}

\begin{document}

\maketitle

O FFmpeg é uma ferramenta poderosa para transcodificação de vídeos e pode ser configurado tanto para usar a CPU quanto a GPU para melhorar o desempenho, especialmente na codificação de vídeos. A seguir, vamos explicar detalhadamente como você pode parametrizar comandos usando FFmpeg, tanto para uso com a GPU (Nvidia CUDA) quanto com a CPU.

\section{Usando a GPU (CUDA) com FFmpeg}

Para aproveitar a aceleração por hardware, o FFmpeg oferece suporte a vários codecs otimizados para GPU, como o \texttt{h264\_nvenc} para H.264 e o \texttt{hevc\_nvenc} para H.265, que usam a tecnologia NVENC das GPUs Nvidia.

\subsection{Parâmetros Principais para GPU}

\begin{itemize}
    \item \texttt{-hwaccel cuda}: Ativa a aceleração por hardware CUDA. Isso indica ao FFmpeg que ele deve usar a GPU para decodificação ou processamento do vídeo.
    
    \item \texttt{-c:v h264\_nvenc} ou \texttt{-c:v hevc\_nvenc}: Define o codec de vídeo NVENC para H.264 ou H.265, que utiliza a GPU para codificação. O NVENC é uma tecnologia de codificação de vídeo acelerada por hardware nas GPUs Nvidia.
    
    \item \texttt{-preset}: Define a velocidade e qualidade da codificação. Com GPUs Nvidia, os valores variam de \texttt{p1} a \texttt{p7} (em versões mais recentes do FFmpeg), sendo:
    \begin{itemize}
        \item \texttt{p1}: Mais rápido (menor qualidade).
        \item \texttt{p7}: Mais lento (maior qualidade).
    \end{itemize}
    
    \item \texttt{-b:v}: Taxa de bits de vídeo. Controla a quantidade de dados alocados para o vídeo por segundo. Um valor mais alto significa melhor qualidade, mas arquivos maiores. Exemplo: \texttt{-b:v 5M} define 5 Mbps.
    
    \item \texttt{-gpu N}: Se você tem várias GPUs, pode especificar qual GPU usar. \texttt{N} é o índice da GPU (começando do zero). Exemplo: \texttt{-gpu 0}.
\end{itemize}

\subsection{Exemplo de Comando Usando GPU}

Este comando usa o codec \texttt{h264\_nvenc} (para H.264), com aceleração CUDA, para codificar um vídeo com legendas:

\begin{lstlisting}
& "C:\caminho\para\ffmpeg.exe" -hwaccel cuda -i "C:\caminho\para\video_input.mp4" -vf "subtitles='C\:\\caminho\\para\\subtitulo.srt'" -c:v h264_nvenc -preset fast -b:v 5M -c:a aac -b:a 192k "C:\caminho\para\video_output.mp4"
\end{lstlisting}

\subsection{Detalhes}

\begin{itemize}
    \item \texttt{-i "C:\caminho\para\video\_input.mp4"}: Arquivo de entrada (vídeo).
    \item \texttt{-vf "subtitles='C:\\caminho\\para\\subtitulo.srt'"}: Adiciona legendas ao vídeo.
    \item \texttt{-c:v h264\_nvenc}: Usa o codec NVENC para H.264.
    \item \texttt{-preset fast}: Usa o preset "rápido" para codificação balanceada entre qualidade e velocidade.
    \item \texttt{-b:v 5M}: Define a taxa de bits de vídeo em 5 Mbps.
    \item \texttt{-c:a aac}: Usa o codec de áudio AAC.
    \item \texttt{-b:a 192k}: Define a taxa de bits do áudio em 192 kbps.
    \item \texttt{-hwaccel cuda}: Ativa a aceleração por hardware CUDA.
\end{itemize}

\section{Usando a CPU com FFmpeg}

Se você deseja usar a CPU para codificação de vídeo, você não precisará de aceleração por hardware. O FFmpeg fornece codecs otimizados para CPU, como o \texttt{libx264} para H.264 e \texttt{libx265} para H.265.

\subsection{Parâmetros Principais para CPU}

\begin{itemize}
    \item \texttt{-c:v libx264} ou \texttt{-c:v libx265}: Define o codec de vídeo que será usado pela CPU. O \texttt{libx264} é amplamente usado para H.264, enquanto o \texttt{libx265} oferece melhor compressão, mas é mais lento.
    
    \item \texttt{-preset}: Semelhante ao uso com GPU, o preset também pode ser aplicado para a CPU. No entanto, os valores de preset para \texttt{libx264} e \texttt{libx265} são diferentes. Eles variam de:
    \begin{itemize}
        \item \texttt{ultrafast}: Codificação mais rápida, menor qualidade.
        \item \texttt{veryfast}, \texttt{faster}, \texttt{fast}: Boa combinação de velocidade e qualidade.
        \item \texttt{medium}: Valor padrão.
        \item \texttt{slow}, \texttt{slower}: Melhor qualidade, mais lento.
        \item \texttt{veryslow}: Máxima qualidade, extremamente lento.
    \end{itemize}
    
    \item \texttt{-crf}: Controle de taxa constante (\textit{Constant Rate Factor}). Um valor mais baixo significa melhor qualidade. O valor recomendado para H.264 é entre \texttt{18} e \texttt{23}. Para H.265, entre \texttt{22} e \texttt{28}.
    
    \item \texttt{-threads N}: Define o número de threads a serem usados pela CPU. Exemplo: \texttt{-threads 8} para usar 8 threads do processador.
\end{itemize}

\subsection{Exemplo de Comando Usando CPU}

Aqui está um comando para codificar um vídeo usando a CPU com o codec \texttt{libx264} (H.264):

\begin{lstlisting}
& "C:\caminho\para\ffmpeg.exe" -i "C:\caminho\para\video_input.mp4" -vf "subtitles='C\:\\caminho\\para\\subtitulo.srt'" -c:v libx264 -preset medium -crf 23 -c:a aac -b:a 192k "C:\caminho\para\video_output.mp4"
\end{lstlisting}

\subsection{Detalhes}

\begin{itemize}
    \item \texttt{-c:v libx264}: Usa o codec \texttt{libx264} para H.264, que roda na CPU.
    \item \texttt{-preset medium}: Usa o preset "medium", que é o valor padrão, balanceando qualidade e velocidade.
    \item \texttt{-crf 23}: Define a qualidade de vídeo. O valor 23 é considerado boa qualidade com tamanho de arquivo razoável. Para uma qualidade melhor, você pode usar \texttt{-crf 18}, mas o arquivo será maior.
    \item \texttt{-c:a aac} e \texttt{-b:a 192k}: Mesma configuração de áudio usada anteriormente.
\end{itemize}

\section{Principais Diferenças entre GPU e CPU}

\begin{itemize}
    \item \textbf{Desempenho}: Usar a GPU pode ser consideravelmente mais rápido para a codificação de vídeo, especialmente com NVENC. No entanto, o controle fino sobre a qualidade (como com \texttt{-crf}) é mais detalhado ao usar a CPU com \texttt{libx264} ou \texttt{libx265}.
    
    \item \textbf{Qualidade vs. Velocidade}: A GPU é ideal para transcodificação rápida, mas a CPU pode oferecer mais flexibilidade e controle sobre a qualidade do vídeo final, especialmente em casos onde a compressão é uma preocupação maior do que o tempo de processamento.
    
    \item \textbf{Taxa de bits (\texttt{-b:v}) vs. CRF (\texttt{-crf})}: Ao usar NVENC (GPU), você geralmente define a taxa de bits manualmente (\texttt{-b:v}), enquanto com a CPU (\texttt{libx264} ou \texttt{libx265}), você pode usar o \texttt{-crf} para ajustar automaticamente a qualidade com base no conteúdo do vídeo.
\end{itemize}

\section{Considerações Adicionais}

\begin{itemize}
    \item \textbf{Tamanho do Arquivo}: Usar \texttt{-crf} em vez de \texttt{-b:v} em codificações com CPU permite que o FFmpeg ajuste a taxa de bits dinamicamente, resultando em tamanhos de arquivo menores sem sacrificar qualidade visível.
    
    \item \textbf{Compatibilidade de Hardware}: Nem todos os sistemas possuem GPUs compatíveis com NVENC. Se o seu hardware não suporta, você precisará usar a CPU.
    
    \item \textbf{Controlando a Resolução}: Você pode redimensionar o vídeo durante a codificação, seja com CPU ou GPU, adicionando o filtro \texttt{scale}. Exemplo para redimensionar para 1280x720:
    \begin{lstlisting}
    -vf "scale=1280:720"
    \end{lstlisting}
\end{itemize}

\section{Conclusão}

\begin{itemize}
    \item \textbf{Com GPU (CUDA)}: Use a aceleração de hardware para ganhos de desempenho significativos, principalmente em transcodificações frequentes ou de longa duração.
    \item \textbf{Com CPU}: Obtenha maior controle sobre a qualidade de vídeo, mas a custo de tempos de processamento mais longos.
\end{itemize}

Ambas as opções têm seus pontos fortes, e a escolha entre GPU e CPU dependerá das necessidades específicas do projeto e do hardware disponível.

\end{document}
